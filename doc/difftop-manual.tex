\documentclass[11pt]{article}
\usepackage{subfigure}
\usepackage{graphics}
\usepackage{amssymb}
\usepackage{epsf}
\usepackage{epsfig}
\usepackage{rotating}
\usepackage{pstricks}
\usepackage{color}
\usepackage{url} 
%csm added
\usepackage{cite} 
%
\usepackage[margin=2.cm]{geometry}
%\topmargin{2.cm}
\newcommand{\lr}[1]{{ \left( \, #1 \, \right) }}
\def\baselinestretch{1.3}
\textheight 23.5cm   \textwidth 16cm
\oddsidemargin -2pt
\evensidemargin 0pt
\newcommand{\lbl}[1]{\label{eq:#1}}
\newcommand{\epo}{\;\:.}
\newcommand{\amu}{a_\mu}
\def\d{\delta}
\def\w{\omega}
\def\nbar{\bar{n}}
\def\slashed{\ds}
\def\e{\epsilon}
\def\a{\alpha}
\def\b{\beta}
\def\bs{\bar\beta}\def\ts{\bar t}
\def\l{\lambda}
\def\im{\imath}
%\def\d{\Delta}
\def\g{\gamma}
\def\dpr{\Delta '}
\def\su{\sigma_1}
\def\sd{\sigma_2}
\def\st{\sigma_3}
\def\lp{\lambda_{+}}
\def\lm{\lambda_{-}}
\def\lpm{\lambda_{\pm}}
\def\taup{\tau_{+}}
\def\taum{\tau_{-}}
\def\taupm{\tau_{\pm}}
%\def\th{\theta}
%\def\t{s_3}
\def\zp{\xi_{+}}
\def\zm{\xi_{-}}
\def\zpm{\xi_{\pm}}
\def\cp{\xi_{+}}
\def\cm{\xi_{-}}
\def\cpm{\xi_{\pm}}
\def\r{\rho}
\def\m{\mu}
\def\rop{\rho_{+}}
\def\rom{\rho_{-}}
\def\ropm{\rho_{\pm}}
\def\di{\eta_-}
\def\so{\eta_+}
\def\tol{\longrightarrow}
\def\li#1{\,{\rm Li}_2\left(#1 \right)}
\def\dl#1{$$\displaylines{\quad#1}$$}
\def\nl{\hfill\cr\hfill}
\def\tr{\mathop{\rm Tr}}
\def\ds#1{#1\kern-1ex\hbox{/}}
\def\dsh{h\kern-1.2ex /}
\def\mod#1{|{\bf #1}|}
\def\Nep{{\rm N}(\e)}


\let\G=\Gamma  \let\D=\Delta \let\Q=\Theta  \let\L=\Lambda
\let\X=\Xi     \let\P=\Pi    \let\S=\Sigma  \let\U=\Upsilon
\let\F=\Phi    \let\Y=\Psi   \let\C=\Chi    \let\W=\Omega
%
\newcommand{\bea}{\begin{eqnarray}}
\newcommand{\eea}{\end{eqnarray}}
\newcommand{\td}{\tilde}
\def\ul{\underline}
\def\nn{\nonumber}
\def\beq{\begin{equation}}
\def\eeq{\end{equation}}
\def\Q{{\bf Q}}
\def\z{{\zeta}}
\def\R{{\bf R}}
\def\beqn{\begin{eqnarray}}
\def\eeqn{\end{eqnarray}}
\def\ba{\begin{eqnarray}}
\def\ea{\end{eqnarray}}
\def\ie{{\it i.e.}}
\def\eg{{\it e.g.}}
\def\half{{ style{1\over 2}}}
\def\nicefrac#1#2{\hbox{${#1\over #2}$}}
\def\third{{ style {1\over3}}}
\def\quarter{{ style {1\over4}}}
\def\m{{\tt -}}
\def\atp{\frac{\alpha_s(Q^2)}{2\pi}}
\def\afp{\frac{\alpha_s(Q^2)}{4\pi}}
\def\p{{\tt +}}

\def\slash#1{#1\hskip-6pt/\hskip6pt}
\def\slk{\slash{k}}
\def\GeV{\,{\rm GeV}}
\def\TeV{\,{\rm TeV}}
\def\y{\,{\rm y}}
\def\one {\bf  1}

\hyphenation{counter-terms}

\setcounter{footnote}{0}
\newcommand{\be}{\begin{equation}}
\newcommand{\beqa}{\begin{eqnarray}}
\newcommand{\eeqa}{\end{eqnarray}}
\newcommand{\eps}{\epsilon}
\newcommand{\la}{\lambda}
\newcommand{\ro}{\rho}
\newcommand{\si}{\sigma}
\newcommand{\veps}{\varepsilon}
\newcommand{\ee}{\end{equation}}
%\makeatletter
%\usepackage{babel}
%\makeatother
\def\li#1{\,{\rm Li}_2\left(#1 \right)}
\newcommand{\Li}[2]{{\mbox{Li}}_{#1}\left(#2\right)}


\begin{document}
%\noindent
%DESY 14-077\\
%LPN 14-076\\
%SFB/CPP-14-27\\
\hfill
\begin{center}

\vspace{1.0cm}
{\bf \large \textsc{DiffTop}: a code for the calculation of the differential cross section for heavy-quark pair production at hadron colliders}

\vspace{1.0cm}

{\bf \large \textsc{DiffTop}-1.0.0 Users manual}
\vspace{0.5cm}

\vspace{1cm}
{ \bf Marco Guzzi$^a$, Katerina Lipka$^a$ and Sven-Olaf Moch$^{b,c}$}


%\vspace{1cm}


{\it $^a$Deutsches Elektronen-Synchrotron DESY, \\Notkestrasse 85, D-22607 Hamburg, Germany\\}
{\it $^b$II.~Institut f\"ur Theoretische Physik, Universit\"at Hamburg, \\Luruper Chaussee 149, D-22761 Hamburg, Germany\\}
{\it $^c$Deutsches Elektronen-Synchrotron DESY,\\ Platanenallee 6, D-15738 Zeuthen, Germany}
\vspace{.5cm}\footnote{marco.guzzi@desy.de, katerina.lipka@desy.de, sven-olaf.moch@desy.de}

\begin{abstract}


The \textsc{DiffTop} Fortran based package is presented. 
This program allows the user to calculate the differential and total cross section 
for heavy-quark pair production at hadron colliders in 1PI kinematics. 
The cross sections are calculated in perturbative 
QCD at leading ${\cal O}(\alpha_s^2)$, approximate next-to-leading ${\cal O}(\alpha_s^3)$ and 
approximate next-to-next-to-leading order ${\cal O}(\alpha_s^4)$, by using 
methods of threshold resummation beyond the leading logarithmic accuracy.
The package is very flexible and all computation inputs can be easily modified by 
editing a user options text file.


\end{abstract}
\end{center}
\date{\today}
\newpage



\section{Program summary}\label{sec:Program summary}

{\it Program title:} \textsc{DiffTop}~(ver. 1.0.1)
\newline
{\it Licensing provisions:} none
\newline
{\it Programming language:} \textsc{Fortran}
\newline
{\it Compiler:} gfortran 
\newline
{\it Computer:} All 
\newline
{\it Operating system:} Linux 
\newline
{\it Running time:} For the differential cross sections it depends on the inputs. 
On a ``Intel Core-i7'' CPU M 640 at 2.80GHz $\times$ 4, a usual run takes 7-8 hours for a 40 points grid. 
The total cross section computation takes approximately 30 minutes. 
\newline
{\it Nature of the problem:} Calculation of differential and total cross section at LO, 
approximate NLO and NNLO in QCD by using techniques of threshold resummation beyond the leading logarithmic approximation.
\newline
{\it Key words:} Top-quark Physics, Hadron Colliders, QCD, Threshold Resummation, 


\section{Introduction}\label{sec:kinem-cross-sect}

Studies of heavy-quark production at hadron colliders provide stringent tests of quantum chromodynamics 
(QCD) and of the theory of electroweak (EW) interactions. These are of crucial 
importance in searches for signatures of physics Beyond the Standard Model (BSM).

The comparison of QCD predictions with the data and a multitude of phenomenological analyses of interest  
require to have precise predictions not only for the inclusive cross section, but also at differential level. 
Invariant mass distribution of the heavy-quark pair, transverse momentum and rapidity distributions of the final-state 
heavy quark or pair, are examples of differential distributions needed at the highest perturbative order possible. 

The QCD corrections to heavy-quark production at hadron colliders at the next-to-leading order 
(NLO), ${\cal O}(\alpha_s^3)$, are known since many years~\cite{Nason:1987xz,Nason:1989zy,Beenakker:1988bq,Meng:1989rp,Beenakker:1990maa,Mangano:1991jk}. 
The full calculation at next-to-next-to-leading order (NNLO), ${\cal O}(\alpha_s^4)$, 
for the inclusive cross section has been accomplished only recently~\cite{Czakon:2013goa,Czakon:2012pz,Czakon:2012zr,Baernreuther:2012ws} 
and required continuous efforts of the QCD community in calculating radiative corrections 
and in the development of computational tools~\cite{Czakon:2007wk,Czakon:2007ej,Mitov:2006xs,Ferroglia:2009ep,Ferroglia:2009ii,Czakon:2010td,Bierenbaum:2011gg,Baernreuther:2013caa}. The NNLO calculation of the inclusive cross section for the $t\bar{t}$ production is implemented 
in the \textsc{C++} computer programs \textsc{Top++}~\cite{Czakon:2011xx} and \textsc{Hathor}~\cite{Aliev:2010zk}.
In these calculations the final-state top quarks are considered in the on-shell approximation.

The use of threshold resummation methods allows one to derive approximate formulas 
at NNLO for differential distributions, in which the cross sections
are expanded in terms of the logarithmic enhanced contributions (appearing as plus distributions), 
and can therefore be written at various degrees of logarithmic accuracy.
Studies at the next-to-leading-logarithmic (NLL), next-to-next-to-leading-logarithmic (NNLL) 
accuracy and beyond can be found in Refs.~\cite{Kidonakis:2014isa,Kidonakis:2000ui,Kidonakis:2001nj,Kidonakis:2003tx,Kidonakis:2003qe,Kidonakis:2005kz,Kidonakis:2008mu,Kidonakis:2010tc,Kidonakis:2010dk,Czakon:2009zw} and references therein.

The approximate NNLO ${\cal O}(\alpha_s^4)$ calculation for the 
differential cross section in the single-particle inclusive (1PI) kinematic for heavy-flavor 
production at hadron colliders is discussed in Ref.~\cite{Guzzi:2014wia}. 
This calculation is implemented in a novel computer code $\textsc{DiffTop}$, that is presented here 
and is available at~\cite{difftop_web} .
Techniques of logarithmic expansion beyond the leading 
logarithmic accuracy in QCD threshold resummation are used.









\section{One-particle inclusive (1PI) kinematics}\label{sec:one-part-incl}

In this section a brief description of the 1PI kinematics is given together with 
the structure of the differential cross section calculation as implemented in the code.
More details can be found in Ref.~\cite{Guzzi:2014wia}.

Heavy-quark hadroproduction in 1PI kinematics is defined by the reaction
\begin{eqnarray}
\label{eq:3}
h_1(P_1) +\, h_2(P_2) &\longrightarrow& {\rm{Q}}(p_1) +\,
X[{\overline{\rm{Q}}}](p_X)\, , 
\end{eqnarray}
where $h_1$ and $h_2$ are hadrons, $X[\overline{\rm{Q}}]$ denotes any
allowed hadronic final state of momentum $p_X$ containing at least the heavy antiquark,
and ${\rm{Q}}(p_1)$ is the identified heavy quark with mass $m$. The
hadronic invariants in this reaction are
\begin{equation}
  \label{eq:5}
S = (P_1+P_2)^2 \quad,\quad T_1 = (P_2-p_1)^2-m^2 \quad,\quad U_1 =
(P_1-p_1)^2-m^2\,, 
\end{equation}
and
\begin{equation}
  \label{eq:6}
S_4 = S+T_1+U_1\,  ,  
\end{equation}
where $S_4$ is a measure of the inelasticity of the hadronic reaction 
(\ref{eq:3}).  Near the threshold, reaction (\ref{eq:3}) is dominated by
the partonic subprocesses
\begin{eqnarray}
  \label{eq:7}
q(k_1) +\, {\overline{q}}(k_2) &\longrightarrow& {\rm{Q}}(p_1) +\,
X'[{\overline{\rm{Q}}}](p_2')\, ,
\\[1ex]
\label{eq:8}  
g(k_1) +\, g(k_2) &\longrightarrow& {\rm{Q}}(p_1) +\,
X'[{\overline{\rm{Q}}}](p_2')\, . 
\end{eqnarray}
If $X'[{\overline{\rm{Q}}}](p_2')={\overline{\rm{Q}}}(\bar{p}_2)$, the reaction
is at partonic threshold and the heavy antiquark has momentum
$\bar{p}_2$.  Note that threshold production does not
mean that the heavy quarks are produced at rest.  The $qg$ and
$\overline{q}g$ channels contribute at one order higher in $\alpha_s$ than 
reactions (\ref{eq:7}) and ({\ref{eq:8})~\footnote{For example,
 for top quark pair production at the Tevatron
  $p\bar{p}$ collider these subprocesses contribute
  only 1\% of the total cross section 
\cite{Nason:1987xz,Beenakker:1988bq,Beenakker:1990maa}.}.  
The partonic invariants corresponding to the hadronic ones in Eq.(\ref{eq:5}) are
\begin{eqnarray}
  \label{eq:9}
s\,=\,(k_1+k_2)^2\, , \hspace*{10mm}t_1\,=\,(k_2-p_1)^2-m^2\, ,
\hspace*{10mm}u_1\,=\,(k_1-p_1)^2-m^2\, .
\end{eqnarray}
The invariant $s_4 = (p'_2)^2-m^2$ which measures the inelasticity of
the partonic reactions (\ref{eq:7}) and (\ref{eq:8}) is related to the
other partonic invariants by
\begin{eqnarray}
  \label{eq:10}
s_4 &=& s + t_1 + u_1\, .  
\end{eqnarray}



The factorized differential cross section for reaction (\ref{eq:3}) is written as 
\ba
S^2\frac{d^2 \sigma(S,T_1,U_1)}{dT_1 ~dU_1} &=& \sum_{i,j=q,\bar{q},g} \int_{x_1^-}^{1} \frac{dx_1}{x_1}
\int_{x_2^-}^{1}\frac{dx_2}{x_2} f_{i/H_1}(x_1,\mu_F^2) f_{j/H_2}(x_2,\mu_F^2) 
\nonumber\\ 
&&\times \, \omega_{ij}(s,t_1,u_1,m_Q^2, \mu_F^2,\alpha_s(\mu_R^2)) + {\cal O}(\Lambda^2/m_Q^2)
\label{diffXsec}
\ea
where $f_{j/H}(x,\mu_F^2)$ is the probability of finding the parton $j$ in hadron $H$, 
$\mu_F$ and $\mu_R$ are the factorization and renormalization scales respectively, and  
$\omega_{ij}$ is the hard scattering cross section in the 1PI kinematic. 
Power suppressed terms indicated by ${\cal O}(\Lambda^2/m_Q^2)$ are neglected in the current implementation.

The parton momentum fractions $x_1$ and $x_2$ have lower limits $x_1^-$ and $x_2^-$, which 
in 1PI kinematics are given by $x_1^- = -U_1/(S+T_1)$ and $x_2^-= -x_1 T_1/(x_1S+U_1)$.
Ref.~\cite{Beenakker:1990maa} contains an exact NLO treatment of this kinematic 
at the parton and hadron levels.

According to QCD resummation, in the organization of the large logarithms 
at the threshold of the heavy system, the hard scattering $\omega_{ij}$ functions are expanded in 
terms of singular functions which are plus-distributions of the type
\beq
\left[\frac{\ln^l{(s_4/m_Q^2)}}{s_4} \right]_{+} = \lim_{\Delta\rightarrow 0}\left\{\frac{\ln^l{(s_4/m_Q^2)}}{s_4}\theta(s_4-\Delta)
+\frac{1}{l+1}\ln^{l+1}{\left( \frac{\Delta}{m_Q^2}\right)}\delta(s_4)\right\}\,,
\label{plus}
\eeq
where corrections are denoted as leading-logarithmic (LL) if $l=2 i+1$ at ${\cal O}(\alpha_s^{i+3})$ with $i=0,1,\dots$, 
as next-to-leading logarithm (NLL) if $l=2 i$, as next-to-next-to-leading logarithm (NNLL) if $l=2 i - 1$, and so on.

The variables $T_1$ and $U_1$ are related in 1PI kinematics to 
the transverse momentum $p_T$ and rapidity $y$ of a single final-state heavy quark.
The definition of the transverse mass $m_T$ of a single heavy quark is given by 
\begin{equation}
\label{eq:mt2-1pi}
  m_T^2 =  m^2 + p_T^2\, ,
\end{equation}
and precise relations to the hadronic variables can be obtained as 
\begin{eqnarray}
T_1 \,=\, - m_T\, \sqrt{S}\,\exp(+y)\, ,  
\qquad\qquad 
U_1 \,=\, - m_T\,\sqrt{S}\,\exp(-y)\, .
\end{eqnarray}

Some algebra leads to the integration boundaries, which are given by  
\begin{eqnarray}
{S \over 2} \big(1 - \beta_S \big)   
\,\le& -T_1 &\le\, 
{S \over 2} \big(1 + \beta_S \big) \, , \\
- m^2 { S \over T_1}
\,\le& -U_1 &\le\, 
S + T_1\, ,
\end{eqnarray}
where $\beta_S=\sqrt{1-4m^2/S}$.
Depending on the order of integration, the phase space boundaries for
$y$ and $m_T$ are as follows.
For 
\begin{eqnarray}
\int d(-T_1) \int d(-U_1) &=& S \int d m_T^2 \int dy\, ,
  \label{eq:mty-1pi}
\end{eqnarray}
one has
\begin{eqnarray}
m^2 
\,\le& m_T^2 &\le\, 
{S \over 4} \, , 
\label{eq:mtybds-1-1pi}
\\
- \cosh^{-1}\bigg({ \sqrt{S} \over 2 m_T} \bigg)
\,\le& y &\le\, 
\cosh^{-1}\bigg({ \sqrt{S} \over 2 m_T} \bigg)\, ,
\label{eq:mtybds-2-1pi}
\end{eqnarray}
and for 
\begin{eqnarray}
\int d(-T_1) \int d(-U_1) &=& S \int d y \int d m_T^2\, ,
  \label{eq:ymt-1pi}
\end{eqnarray}
one has
\begin{eqnarray}
- {1 \over 2} \ln\bigg({1 + \beta_S \over 1 - \beta_S} \bigg)
\,\le& y &\le\, 
{1 \over 2} \ln\bigg({1 + \beta_S \over 1 - \beta_S} \bigg) \, ,
\label{eq:ymtbds-1-1pi}
\\
m^2 \,\le& m_T^2 &\le\, 
{S \over 4 \cosh^2(y)} \, .
\label{eq:ymtbds-2-1pi}
\end{eqnarray}





\section{The \textsc{DiffTop} program}\label{sec:The DiffTop program}


\textsc{DiffTop} is a Mellin space based threshold resummation code for the calculation of the 
differential and total cross section for heavy-quark pair production 
in 1PI kinematics at hadron colliders.
Depending upon the choice of the inputs selected by the user through a user option file, 
\textsc{DiffTop} calculates cross sections
at LO ${\cal O}(\alpha_s^2)$, approximate NLO ${\cal O}(\alpha_s^3)$ and NNLO ${\cal O}(\alpha_s^4)$,
in perturbative QCD beyond the leading logarithmic accuracy (NLL, NNLL).

A selection of the desired parton distribution functions (PDFs) and evolution 
of the strong coupling constant $\alpha_s$ is made through an interface to 
the modern LHAPDF libraries~\cite{Butterworth:2014efa,lhapdf_web}.

\textsc{DiffTop} is a tool for phenomenological analyses of heavy-flavor production at hadron colliders, 
in particular for top-quark pair production. 
Physics examples and detailed phenomenological studies can be found in Ref.~\cite{Guzzi:2014wia},
where the results of transverse momentum and rapidity distributions 
of final-state top quarks in proton-proton collisions at the LHC are compared 
to the recent measurements by the ATLAS and CMS collaborations 
at the center of mass energy of 7 TeV.  

This program provides a basis for the inclusion of differential 
$t\bar{t}$ cross sections into QCD analyses for the determination of PDFs at NNLO. 
For the purpose of a fast calculation within global QCD analyses, 
\textsc{DiffTop} is interfaced to \textsc{fastNLO}~\cite{dis2014Fast,Britzger:2012bs,Wobisch:2011ij,Kluge:2006xs} 
and included into the QCD analysis platform \textsc{HERAFitter}~\cite{herafitter}. 
The \textsc{DiffTop} interface to \textsc{fastNLO} allows, for the first time, the \textsc{HERAFitter} user 
to perform full PDF fits by using the differential cross sections 
of $t\bar{t}$ production. This allows the users to fully explore the potential 
of the recent $t\bar{t}$ measurements at the LHC in QCD analyses for PDF determination.







\section{Program installation}\label{sec:Program installation}

\subsection{Content of the package}\label{sec:Content of the package}

The package can be downloaded at \cite{difftop_web} and it requires the installation of the 
LHAPDF libraries which are available at \cite{lhapdf_web}.
The presented version of the package is tested with LHAPDF versions 5.8.9 and 5.9.0.

The user can extract the files from the .tar.gz package by running the command:
\begin{verbatim}
tar -zxvf difftop-1.0.0.tar.gz
\end{verbatim}
This creates the directory \texttt{difftop-1.0.0} containing the subdirectory \texttt{manual} which includes the 
users manual, and the following files:

\begin{itemize}

\item \texttt{setup.sh}: a small \texttt{shell} script containing lines to be edited by 
the user for the correct inclusion of paths for the executables and LHAPDF libraries.

\item \texttt{Makefile}: it allows the user to compile and link the Fortran programs with the \texttt{gfortran} compiler. 
It might be edited to change the path for the LHAPDF libraries. The default option uses \texttt{lhapdf-config}:
\begin{verbatim}
LHAPDFLIBPATH = $(shell lhapdf-config --prefix)/lib
\end{verbatim}

\item \texttt{difftop1PI.f}: is the main file. It includes the transverse momentum $p_T$ and rapidity $y$ grids generation, 
and the calls to the \textsc{Vegas} algorithm (documented in Ref.~\cite{Lepage:1977sw}) integration routine. 

\item \texttt{Sfuncts.f}: it includes the analytical structure of the hard and soft functions of the threshold resummation formalism.

\item \texttt{integrandpty.f}: it includes the logarithmic expansion and it represents the core of the calculation.

\item \texttt{hadfold.f}: it includes the calls to the PDFs and $\alpha_s(\mu_R)$. 
It includes the calls to initialize 
the LHAPDF libraries and can be also customized for an independent stand-alone PDF module.

\item \texttt{pac1.f}: is the wrapper for the LHAPDF libraries.

\item \texttt{difftop.in}: is the user-option input file containing 29 options. It needs to be edited 
according to the requirements of the specific computation.  

\item \texttt{wgplg.f}: auxiliary Fortran subroutine for the calculation 
of polylogarithmic functions available at~\cite{wgplg}.

\item \texttt{vegas.f}: is a Fortran subroutine including the \textsc{Vegas} 
algorithm~\cite{Lepage:1977sw,Lepage:1980dq,Press:1992zz}.

 
\end{itemize}



\subsection{Compiling}\label{sec:Compiling}

Once the correct paths for the LHAPDF libraries are included in the \texttt{shell} script \texttt{setup.sh},  
the user can source it by running the command:
\begin{verbatim}
source setup.sh
\end{verbatim}

Then, for compiling and linking, it is only necessary to type: 
\begin{verbatim}
make
\end{verbatim}

The executable file is generated, that is called \texttt{difftop1PI}.


\section{Running \textsc{DiffTop}}\label{sec:Running DiffTop}

To run \textsc{DiffTop}, in the current directory \texttt{./difftop-1.0.0},
the user must type the command:
\begin{verbatim}
difftop1PI
\end{verbatim}
This starts the evaluation and the value(s) of the total(differential) cross section 
are written in the output file \texttt{difftop.out} once the calculation is terminated.
The file \texttt{difftop.out} is explained in the next sections.


\subsection{Description of the input options file \texttt{difftop.in}}\label{sec:Description of the input options}

Before starting the evaluation the user must edit the text file \texttt{difftop.in} according 
to the configuration of the required calculation.
The file \texttt{difftop.in} contains 29 lines arrenged in two columns (left and right). 
The various options (left column) are decided by assigning a numerical (integer or real) value to the variables.
A string of characters is utilized to decide which heavy quark is produced at the chosen collider.
The right column contains definitions with a short explanation of the variables. 
The structure of the file is illustrated in Table~\ref{difftop.in} that is explained line-by-line in what follows.
%%%%%%%%%%%%%%%%%%%%%%%%%%%%
\begin{table}[ht]
%\hspace{-6.0cm}
\begin{centering}
\begin{scriptsize}
\begin{tabular}{|c|c|c|}
\hline 
\multicolumn{3}{|c|}{Description of the input file \texttt{difftop.in}} \tabularnewline
\hline 
\hline 
1  &     0         &       choice: $p_T$ (0) or $y$ (1) distrib.; $\sigma^{p_T}_{TOT}$ (2) or $\sigma_{TOT}^y$ (3); \tabularnewline
	\hline
2  &     1         &       \texttt{PDFflag}: (1) LHAPDF; (0) stdalone PDFs module;\tabularnewline
	\hline
3  &     11200     &       \texttt{Iset}: index to select the PDFs relative to .LHgrid;\tabularnewline
	\hline
4  &     99        &       \texttt{nalfas}: (99) LHAPDF $\alpha_s$; (2,3) $\Lambda_{QCD}$ version, $\alpha_s(M_Z) = 0.118$;\tabularnewline 
	\hline
5  &     0.0878    &       \texttt{qcdl1} for $n^l_{f}=5$;\tabularnewline
	\hline
6  &     0.226     &       \texttt{qcdl2} for $n^l_{f}=5$; \tabularnewline
	\hline
7  &     0.186     &       \texttt{qcdl3} for $n^l_{f}=5$; \tabularnewline
	\hline
8  & 	0         & 	  \texttt{ipdf}: is (0) when \texttt{PDFflag}=1$\rightarrow$LHAPDF; (1) stdalone PDFs;\tabularnewline 
	\hline
9  & 	2         & 	  \texttt{naccur}: (0)  LL, (1) NLL, (2) NNLL;\tabularnewline
	\hline
10  &  	1         &  	  \texttt{nggborn}: $GG$ channel Born level;\tabularnewline
	\hline
11  & 	1         & 	  \texttt{nqqborn}: $Q\overline{Q}$ channel Born level;\tabularnewline
	\hline
12  & 	1         & 	  \texttt{nggnlo}: $GG$ channel NLO;\tabularnewline
	\hline
13  & 	1         & 	  \texttt{nqqnlo}: $Q\overline{Q}$ channel NLO;\tabularnewline
	\hline
14  & 	1         & 	  \texttt{nggnnlo}: $GG$ channel NNLO;\tabularnewline
	\hline
15  & 	1         & 	  \texttt{nqqnnlo}: $Q\overline{Q}$ channel NNLO;\tabularnewline
	\hline
16  & 	0         & 	  scale choice: (0) scale = $m_Q$; (1) scale $= \sqrt{m_Q^2+p_T^2}$\tabularnewline
	\hline
17  & 	1.0d0     & 	  $k_F=\mu_F/$scale \tabularnewline
	\hline
18  & 	1.0d0     & 	  $k_R=\mu_R/$scale \tabularnewline
	\hline
19  &     173.0d0   &       $m_q$: mass of the heavy quark \tabularnewline
	\hline
20  &     7000.0d0  &       center of mass energy [GeV]\tabularnewline
	\hline
21  &     5         &       Number of light flavors $n^l_{f}$ \tabularnewline
	\hline
22  &     top@LHC   &       heavy-quark produced at collider \tabularnewline
	\hline
23  &     1.0d0     &       $p_{T,min}$: Min val in $p_T$ grid \tabularnewline
	\hline
24  &     80.80d0   &       $p_{T,max}$: Max val in $p_T$ grid \tabularnewline
	\hline
25  &     40        &       $N_{pt}$: Num points for $(N_{pt}+1)$ $p_T$-output grid \tabularnewline 
	\hline
26  &     -3.5d0    &       $y_{min}$: Min val in $y$ grid \tabularnewline
	\hline
27  &     3.5d0     &       $y_{max}$: Max val in $y$ grid \tabularnewline
	\hline
28  &     40        &       $N_y$: Num points for $(N_y+1)$ $y$-output grid \tabularnewline
	\hline
29  &     0.2d0     &       cut-off parameter \texttt{del} \tabularnewline
	\hline 
\end{tabular}
\end{scriptsize}
\par
\end{centering}
\caption{Example of \texttt{difftop.in}. Input options for the \textsc{DiffTop} calculation 
at approx NNLO + NNLL for calculating the $p_T$ spectrum 
of the final-state top quark at the LHC 7 TeV.}
\label{difftop.in}
\end{table}


\begin{itemize}

\item {\bf line 1.} Integer number that selects the type of cross section which is being computed: 
\\
0 - transverse momentum $d\sigma/p_T$ distribution of the final-state heavy quark, in [pb/GeV];
\\
1 - rapidity distribution $d\sigma/dy$ in [pb];   
\\
2 - total cross section in [pb] obtained by the integral of the $p_T$ distribution;
\\
3 - total cross section in [pb] obtained by the integral of the $y$ distribution.

\item {\bf line 2.} Integer number that selects between the LHAPDF libraries and the stand-alone PDFs:
\\
0 - stand-alone PDFs ;
\\ 
1 - PDFs are taken from the LHAPDF libraries. 

\item {\bf line 3.} Integer number that selects the index of the PDF set relative to \texttt{.LHgrid} ({\it e.g.}  11200 is used for CT10NNLO, 21200 MSTW08, etc.).



\item {\bf line 4.} Integer number that selects the $\alpha_s(\mu_R)$ running:
\\
99 - default value for LHAPDF based $\alpha_s$ evolution;
\\
(1,2,3) - value for the parametric $\alpha_s$ evolution in terms of the 
asymptotic scale parameter $\Lambda$ at (1,2,3)-loop~\footnote{The implementation of the parametric formula 
is cross checked against Eq.(5) of Ref.~\cite{Chetyrkin:2000yt}}.



\item {\bf line 5.} Real number that is set to the value of $\Lambda$ at 1-loop in the $\overline{\textrm{MS}}$. 
The default input is computed by using the \texttt{RunDec} code~\cite{Chetyrkin:2000yt} with input $\alpha_s(M_Z)=0.118$ and $n_f=5$.  


\item {\bf line 6.} The same as line 5 but at 2-loop. 


\item {\bf line 7.} The same as line 5 but at 3-loop.


\item {\bf line 8.} Integer number that acts as a control flag in order to avoid 
inconsistent evolutions of PDFs and $\alpha_s$: 
\\
0 - when the LHAPDF libraries are used at line 2, (default choice for most of the applications);   
\\
1 - in case stand-alone PDFs are used at line 2. 



\item {\bf line 9.} Integer number that select the logarithmic accuracy:
\\
0 - LL;
\\
1 - NLL;
\\
2 - NNLL.

\item {\bf line 10.} Integer number to activate the gluon-gluon channel at the Born level:
\\
(0,1) - (GG Born not active, GG Born active).
 
\item {\bf line 11.} Integer number to activate the quark-antiquark channel at the Born level:
\\
(0,1) - (Q$\overline{\textrm{Q}}$ Born not active, Q$\overline{\textrm{Q}}$ Born active).

\item {\bf line 12.} Integer number to activate the gluon-gluon channel at NLO:
\\
(0,1) - (GG@NLO not active, GG@NLO active).
 
\item {\bf line 13.} Integer number to activate the quark-antiquark channel at NLO:
\\
(0,1) - (Q$\overline{\textrm{Q}}$@NLO not active, Q$\overline{\textrm{Q}}$@NLO active).

\item {\bf line 14.} Integer number to activate the gluon-gluon channel at NNLO:
\\
(0,1) - (GG@NNLO not active, GG@NNLO active).
 
\item {\bf line 15.} Integer number to activate the quark-antiquark channel at NNLO:
\\
(0,1) - (Q$\overline{\textrm{Q}}$@NNLO not active, Q$\overline{\textrm{Q}}$@NNLO active).

\item {\bf line 16.} Integer number to select the scale choice in the calculation:
\\
0 - the scale is set to the mass of the heavy quark $Q=m_Q$ (default choice);
\\
1 - the scale is set dynamically according to $Q=\sqrt{m_Q^2 + p_T^2}$;

\item {\bf line 17.} Real number to define the value of $k_F=\mu_F/scale$ for the factorization scale $\mu_F$ ($k_F=1$ default).

\item {\bf line 18.} Real number to define the value of $k_R=\mu_R/scale$ for the renormalization scale $\mu_R$ ($k_R=1$ default).

\item {\bf line 19.} Real number to define the value of heavy-quark mass. It is interpreted as the pole-mass.

\item {\bf line 20.} Real number to define the value of the center of mass energy $\sqrt{S}$ in GeV.

\item {\bf line 21.} Integer number to define the number of light flavors 
in the calculation (\texttt{nlf}=5 is the default choice for top-quark pair production).

\item {\bf line 22.} String of characters to define the heavy-quark produced 
at the collider ({\it e.g.} \texttt{top@LHC} for top-quark pair production at the LHC, 
\texttt{bottom@TEV} for the bottom-quark case at the Tevatron, etc.).

\item {\bf line 23.} Real number to set the minimum value of the transverse momentum $p_{T,min}$. 
By default, the minimum value is never taken below 1 GeV. This option holds at differential level only.

\item {\bf line 24.} The same as line 23, but for the maximum value of the transverse momentum $p_{T,max}$.
 
\item {\bf line 25.} Integer number to define the number of points $N_{pt}$ for the $p_T$ grid. 
The output grid will contain $N_{pt}+1$ points. The analytic expression defining the $p_T$ grid 
can be modified by manipulating the main file \texttt{difftop1PI.f} where the string ``\texttt{loop over pt}'' is located.  

\item {\bf line 26.} Real number to set the minimum value of the rapidity $y_{min}$. This option holds at differential level only.

\item {\bf line 27.} The same as line 26, but for the maximum value of the rapidity $y_{max}$. 

\item {\bf line 28.} The same as line 25, but for the rapidity $y$. The string to be searched 
in case of manipulation is ``\texttt{loop over y}''.  

\item {\bf line 29.} Real number to set the value of the cutoff variable \texttt{del}. The default value is set to \texttt{del}=0.2.
The calculation is very stable when this value is varied. This value corresponds to the parameter $\Delta$ introduced 
in the plus-distribution definition of Eq.~(\ref{plus}).
\end{itemize}



It important to notice that the calculation of the total cross section, 
when selected at line 1 of \texttt{difftop.in}, 
does not depend on the choices at lines 23, 24 and 26, 27.
These limits on $p_T$ and $y$ are only used for the case 
of differential distribution evaluation (values 0 and 1 at line 1).

According to the current configuration of \textsc{DiffTop}, the LHAPDF and stand-alone $\alpha_s$ and PDFs determinations are kept 
separated by the use of the flags \texttt{PDFflag} and \texttt{ipdf}. It is possible to change this configuration 
by modifying the logical statements located in the Fortran files \texttt{hadfold.f} (where the stand-alone PDFs module 
must be placed) and \texttt{difftop1PI.f}.

To perform a study of renormalization and factorization scale variation 
the user must select the values at lines 17 and 18. 
The case corresponding to $1/2 m_Q \leq \mu_F = \mu_R = m_Q \leq 2 m_Q $ 
is reproduced by launching three separate jobs with the following settings:
\begin{verbatim}
              Run 1          Run 2         Run 3
            ----------------------------------------
              scale = mQ   scale = mQ   scale = mQ     
              k_F = 0.5     k_F = 1.      k_F = 2. 
              k_R = 0.5     k_R = 1.      k_R = 2.
\end{verbatim}
The column relative to Run 2 corresponds to the central configuration. 

Please note that the current \textsc{DiffTop}-1.0.0 version implies only 
simultaneous variations of the renormalization and factorization scales.
The next release of the code will include the necessary terms 
to allow for independent scale variations.
Currently, the scale variation is restricted to $1/4 m_Q \leq \mu_F=\mu_R \leq 4 m_Q$.









\subsection{Description of the output}\label{sec:Description of the output}

The result of the calculation is printed in the text file \texttt{difftop.out}. 
For the case of the total cross section, only one line is printed in which numbers are placed in columns: 
\begin{verbatim}
sqrt(S) [GeV], sigmatot [pb], +- error
\end{verbatim}
The first column contains the input center of mass energy in GeV, 
the second the value of the total cross section in [pb] evaluated at the 
heavy-quark mass $m_Q$ (pole) and scale $\mu_F,\mu_R$,
while the third one contains the error estimate of the \textsc{Vegas} algorithm 
which is given in terms of standard deviation.

For the case of differential distributions, the output is similar:  
\begin{verbatim}
pT [GeV] or y, dsigma/dpT [pb/GeV] or dsigma/dy [pb], +- error
...      ....                                   ....
...      ....                                   .... 
\end{verbatim}
where the first column accounts for the grid of the selected differential variable $p_T$ or $y$,
the second contains the differential distribution in the appropriate units ([pb/GeV] for $p_T$ spectrum 
or [pb] for rapidity), and the third one is the \textsc{Vegas} error.  

The final cross section is obtained as the sum of the Born level, 
inelastic and elastic cross section contributions. 
The inelastic contribution includes terms which are proportional 
to the first logarithm on the right-hand side of Eq.~(\ref{plus}), while the 
elastic one is proportional to logarithms of $\Delta/m_Q^2$ on the left-hand side.

According to the current settings of the integration routines, 
the accuracy of the calculation
\footnote{It refers to the value of the standard deviation output of the \texttt{vegas.f} routine}
for each single contribution is at the per-mill level, 
which is accurate enough for phenomenological applications at collders. 
The settings for the integration accuracy can be modified by searching for the string 
\texttt{VEGAS parameters} in the main file \texttt{difftop1PI.f}.
These modifications will affect the evaluation speed.








\subsection{Test run}\label{sec:Test run}

As an example, the computation for few points of the $p_T$ spectrum 
relative to the input reported in Table~\ref{difftop.in} 
is performed and numerical results are illustrated in Table~\ref{test-run}.
Reported is also the total cross section.

\begin{table}[h]
\begin{centering}
\begin{scriptsize}
\begin{tabular}{|c|c|c|}
\hline 
\multicolumn{2}{|c|}{CT10NNLO,~ LHC 7 TeV,~ $m_t=173$ GeV} \tabularnewline
\hline 
\hline 
$p_T^{t}$ [GeV] & $d\sigma/dp_T$ [pb/GeV] \tabularnewline
\hline 
1  & 0.0437   \tabularnewline
\hline
75  & 1.270   \tabularnewline
\hline 
80  & 1.256   \tabularnewline
\hline 
150  & 0.679   \tabularnewline
\hline 
$\sigma_{tot}$ [pb] & 179.9 \tabularnewline
\hline 
\end{tabular}
\end{scriptsize}
\par
\end{centering}
\caption{A test run with \textsc{DiffTop} corresponding to the input file in Table~\ref{difftop.in} }
\label{test-run}
\end{table}












\section{Summary}\label{sec:Summary}

A description of the \textsc{DiffTop} Fortran based package has been presented.
This code calculates the total and differential cross sections 
for heavy-quark pair production 
at hadron colliders at LO, approximate NLO, and NNLO in QCD, 
by using technique of threshold resummation beyond the leading logarithmic approximation.
This flexible program represents an important tool for
analyses that are of interest for phenomenological 
studies at hadron colliders with particular emphasis on top-quark pair production. 
Details of the distributions of the transverse momentum and rapidity of the 
final-state top-quark can be generated by selecting the desired PDF set, scale, $\alpha_s$ 
and top quark mass $m_t$. 
\textsc{DiffTop} theoretical predictions have been compared to the recent measurements 
by the CMS and ATLAS collaborations, and detailed results together with an 
extensive discussion can be found in Ref.~\cite{Guzzi:2014wia}.




\section*{Acknowledgments}

We are grateful to Daniel Britzger (\textsc{FastNLO} Collaboration) for assistance 
in the implementation of \textsc{DiffTop} in \textsc{FastNLO}, 
and for providing us with the corresponding tables.
This work is supported in part by the 
``Initiative and Networking Fund of the Helmholtz Association (HGF) under the contract S0-072'',
by Deutsche Forschungsgemeinschaft in Sonderforschungs\-be\-reich/Transregio~9, 
by the European Commission through contract PITN-GA-2010-264564 ({\it LHCPhenoNet}), 
and it was realized within the scope of the PROSA collaboration.


\bibliographystyle{h-elsevier3}
\begin{thebibliography}{10}

\bibitem{Nason:1987xz}
P. Nason, S. Dawson and R.K. Ellis,
\newblock Nucl.Phys. B303 (1988) 607.
%%CITATION = NUPHA,B303,607;%%

\bibitem{Nason:1989zy}
P. Nason, S. Dawson and R.K. Ellis,
\newblock Nucl.Phys. B327 (1989) 49.
%%CITATION = NUPHA,B327,49;%%

\bibitem{Beenakker:1988bq}
W. Beenakker et~al.,
\newblock Phys.Rev. D40 (1989) 54.
%%CITATION = PHRVA,D40,54;%%

\bibitem{Meng:1989rp}
R. Meng et~al.,
\newblock Nucl.Phys. B339 (1990) 325.
%%CITATION = NUPHA,B339,325;%%

\bibitem{Beenakker:1990maa}
W. Beenakker et~al.,
\newblock Nucl.Phys. B351 (1991) 507.
%%CITATION = NUPHA,B351,507;%%

\bibitem{Mangano:1991jk}
M.L. Mangano, P. Nason and G. Ridolfi,
\newblock Nucl.Phys. B373 (1992) 295.
%%CITATION = NUPHA,B373,295;%%

\bibitem{Czakon:2013goa}
M. Czakon, P. Fiedler and A. Mitov,
\newblock Phys.Rev.Lett. 110 (2013) 252004, 1303.6254.
%%CITATION = ARXIV:1303.6254;%%

\bibitem{Czakon:2012pz}
M. Czakon and A. Mitov,
\newblock JHEP 1301 (2013) 080, 1210.6832.
%%CITATION = ARXIV:1210.6832;%%

\bibitem{Czakon:2012zr}
M. Czakon and A. Mitov,
\newblock JHEP 1212 (2012) 054, 1207.0236.
%%CITATION = ARXIV:1207.0236;%%

\bibitem{Baernreuther:2012ws}
P. B{\"a}rnreuther, M. Czakon and A. Mitov,
\newblock Phys.Rev.Lett. 109 (2012) 132001, 1204.5201.
%%CITATION = ARXIV:1204.5201;%%

\bibitem{Czakon:2007wk}
M. Czakon, A. Mitov and S. Moch,
\newblock Nucl.Phys. B798 (2008) 210, 0707.4139.
%%CITATION = ARXIV:0707.4139;%%

\bibitem{Czakon:2007ej}
M. Czakon, A. Mitov and S. Moch,
\newblock Phys.Lett. B651 (2007) 147, 0705.1975.
%%CITATION = ARXIV:0705.1975;%%

\bibitem{Mitov:2006xs}
A. Mitov and S. Moch,
\newblock JHEP 0705 (2007) 001, hep-ph/0612149.
%%CITATION = HEP-PH/0612149;%%

\bibitem{Ferroglia:2009ep}
A. Ferroglia et~al.,
\newblock Phys.Rev.Lett. 103 (2009) 201601, 0907.4791.
%%CITATION = ARXIV:0907.4791;%%

\bibitem{Ferroglia:2009ii}
A. Ferroglia et~al.,
\newblock JHEP 0911 (2009) 062, 0908.3676.
%%CITATION = ARXIV:0908.3676;%%

\bibitem{Czakon:2010td}
M. Czakon,
\newblock Phys.Lett. B693 (2010) 259, 1005.0274.
%%CITATION = ARXIV:1005.0274;%%

\bibitem{Bierenbaum:2011gg}
I. Bierenbaum, M. Czakon and A. Mitov,
\newblock Nucl.Phys. B856 (2012) 228, 1107.4384.
%%CITATION = ARXIV:1107.4384;%%

\bibitem{Baernreuther:2013caa}
P. B{\"a}rnreuther, M. Czakon and P. Fiedler,
\newblock JHEP 1402 (2014) 078, 1312.6279.
%%CITATION = ARXIV:1312.6279;%%

\bibitem{Czakon:2011xx}
M. Czakon and A. Mitov,
\newblock (2011), 1112.5675.
%%CITATION = ARXIV:1112.5675;%%

\bibitem{Aliev:2010zk}
M. Aliev et~al.,
\newblock Comput.Phys.Commun. 182 (2011) 1034, 1007.1327.
%%CITATION = ARXIV:1007.1327;%%

\bibitem{Kidonakis:2014isa}
N. Kidonakis,
\newblock Phys.Rev. D90 (2014) 014006, 1405.7046.
%%CITATION = ARXIV:1405.7046;%%

\bibitem{Kidonakis:2000ui}
N. Kidonakis,
\newblock Phys.Rev. D64 (2001) 014009, hep-ph/0010002.
%%CITATION = HEP-PH/0010002;%%

\bibitem{Kidonakis:2001nj}
N. Kidonakis et~al.,
\newblock Phys.Rev. D64 (2001) 114001, hep-ph/0105041.
%%CITATION = HEP-PH/0105041;%%

\bibitem{Kidonakis:2003tx}
N. Kidonakis,
\newblock Int.J.Mod.Phys. A19 (2004) 1793, hep-ph/0303186.
%%CITATION = HEP-PH/0303186;%%

\bibitem{Kidonakis:2003qe}
N. Kidonakis and R. Vogt,
\newblock Phys.Rev. D68 (2003) 114014, hep-ph/0308222.
%%CITATION = HEP-PH/0308222;%%

\bibitem{Kidonakis:2005kz}
N. Kidonakis,
\newblock Phys.Rev. D73 (2006) 034001, hep-ph/0509079.
%%CITATION = HEP-PH/0509079;%%

\bibitem{Kidonakis:2008mu}
N. Kidonakis and R. Vogt,
\newblock Phys.Rev. D78 (2008) 074005, 0805.3844.
%%CITATION = ARXIV:0805.3844;%%

\bibitem{Kidonakis:2010tc}
N. Kidonakis,
\newblock Phys.Rev. D81 (2010) 054028, 1001.5034.
%%CITATION = ARXIV:1001.5034;%%

\bibitem{Kidonakis:2010dk}
N. Kidonakis,
\newblock Phys.Rev. D82 (2010) 114030, 1009.4935.
%%CITATION = ARXIV:1009.4935;%%

\bibitem{Czakon:2009zw}
M. Czakon, A. Mitov and G.F. Sterman,
\newblock Phys.Rev. D80 (2009) 074017, 0907.1790.
%%CITATION = ARXIV:0907.1790;%%

\bibitem{Guzzi:2014wia}
M. Guzzi, K. Lipka and S.O. Moch,
\newblock (2014), 1406.0386.
%%CITATION = ARXIV:1406.0386;%%

\bibitem{difftop_web}
M. Guzzi, K. Lipka and S. Moch,
\newblock 2014,
\newblock {\url{https://difftop.hepforge.org/}}.

\bibitem{Butterworth:2014efa}
J. Butterworth et~al.,
\newblock (2014), 1405.1067.
%%CITATION = ARXIV:1405.1067;%%

\bibitem{lhapdf_web}
The LHAPDF Working Group, A. Buckley et~al.,
\newblock {Forthcoming paper}, 2014,
\newblock {\url{https://lhapdf.hepforge.org/}}.

\bibitem{dis2014Fast}
D. Britzger et~al.,
\newblock {DIS 2014 - XXII. International Workshop on Deep-Inelastic Scattering
  and Related Subjects}, 2014,
\newblock
  {\url{http://indico.cern.ch/event/258017/session/1/contribution/202}}.

\bibitem{Britzger:2012bs}
fastNLO Collaboration, D. Britzger et~al.,
\newblock (2012) 217, 1208.3641.
%%CITATION = ARXIV:1208.3641;%%

\bibitem{Wobisch:2011ij}
fastNLO Collaboration, M. Wobisch et~al.,
\newblock (2011), 1109.1310.
%%CITATION = ARXIV:1109.1310;%%

\bibitem{Kluge:2006xs}
T. Kluge, K. Rabbertz and M. Wobisch,
\newblock (2006) 483, hep-ph/0609285.
%%CITATION = HEP-PH/0609285;%%

\bibitem{herafitter}
The HERAFITTER Working Group, A. Glazov et~al.,
\newblock {Forthcoming paper}, 2014,
\newblock {\url{https://wiki-zeuthen.desy.de/HERAFitter}}.

\bibitem{wgplg}
W.v. Neerven,
\newblock
  {\url{http://www.lorentz.leidenuniv.nl/research/neerven/DECEASED/zwprod.uu}}.

\bibitem{Lepage:1977sw}
G.P. Lepage,
\newblock J.Comput.Phys. 27 (1978) 192.
%%CITATION = JCTPA,27,192;%%

\bibitem{Lepage:1980dq}
G.P. Lepage,
\newblock (1980).
%%CITATION = CLNS-80/447 ETC.;%%

\bibitem{Press:1992zz}
W.H. Press et~al.,
\newblock (1992).
%%CITATION = ISBN-9780521430647 ETC.;%%

\bibitem{Lai:1999wy}
CTEQ Collaboration, H. Lai et~al.,
\newblock Eur.Phys.J. C12 (2000) 375, hep-ph/9903282.
%%CITATION = HEP-PH/9903282;%%

\bibitem{Chetyrkin:2000yt}
K. Chetyrkin, J.H. Kuhn and M. Steinhauser,
\newblock Comput.Phys.Commun. 133 (2000) 43, hep-ph/0004189.
%%CITATION = HEP-PH/0004189;%%

\end{thebibliography}
%\bibliography{top-draft}


\end{document} 




